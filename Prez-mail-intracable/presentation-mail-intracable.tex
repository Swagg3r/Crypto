\documentclass{beamer}

\usepackage[frenchb]{babel}
\usepackage[T1]{fontenc}
\usepackage[utf8]{inputenc}

\usetheme{Warsaw}

\title{Untraceable Electronic Mail, Return Addresses, and Digital Pseudonyms - By David Chaum}
\author{Présenté par Aurélien Monnet-Paquet}
\institute{www.inria.fr}
\date{19 Mai 2016}

\begin{document}

%Premiere page : page de garde.
\begin{frame}
\titlepage
\end{frame}

%Deuxieme page : Introduction
\section{Introduction}
\subsection{L'analyse du trafic réseaux}
\begin{frame}
\frametitle{Problème ?}
\begin{itemize}
\setbeamertemplate{itemize item}[triangle]
\item Comment garder confidentiel qui parle avec qui et quand ?
\pause
\item Cryptographie à clé publique
\end{itemize}
\end{frame}

%Page additionnel : Notation
\subsection{Notation utilisée}
\begin{frame}
\frametitle{Notation}
\begin{itemize}
\setbeamertemplate{itemize item}[triangle]
\item K $~~~~~~~$: Clé publique
\item Inv(K) $~$: Clé privée
\item X $~~~~~~~$: Message en clair
\pause
\item $Inv(K)( K( X ) ) = K( Inv(K)( X ) ) = X$
\pause
\item R $~~~~~~~$: Une chaîne de bits aléatoire
\pause
\item Le chiffrement du message par la clé publique :
\begin{center}
\item $K( R, X )$
\end{center}
\end{itemize}
\end{frame}

%Troisieme page : Hypothèses
\subsection{Hypothèses}
\begin{frame}
\frametitle{Hypothèses}
%\setbeamertemplate{itemize item}[triangle]
%\item Apporter de nouvelles technologies aux IDS :
%\item Support d'IPv6 natif
\begin{block}{}
Hypothèse 1
\end{block}
\begin{block}{}
Hypothèse 2
\end{block}
\end{frame}

%4 eme page : Les projets similaires
\subsection{Les projets similaires}
\begin{frame}
\frametitle{Snort et Bro}
\begin{itemize}
\setbeamertemplate{itemize item}[triangle]
\item Snort
\begin{itemize}
\item Développé par Sourcefire
\item Fonctionnellement équivalent à Suricata
\item Compatibilité Snort / Suricata
\item Concurrence directe
\end{itemize}
\pause
\item Bro
\begin{itemize}
\item Orientation capture
\item Études statistiques
\end{itemize}
\end{itemize}
\end{frame}

%Partie II : Fonctionnement
%5 eme page : Fonctionnement en interne.
\section{Fonctionnement}
\subsection{En interne}
\begin{frame}
\frametitle{Fonctionnement}
\begin{itemize}
\setbeamertemplate{itemize item}[triangle]
\item Lève une alerte mais ne bloque pas le flux (rôle de l'IPS)
\item Travail avec un flux de données
\item Reconstruction du flux : TCP => perte/renvoi/ordre
\item La réception d'un ACK déclenche l'analyse des données.
\end{itemize}
\begin{center}
\end{center}
\end{frame}

%6 eme page : Exemple de regles (Actions)
\subsection{Exemple de règles}
\begin{frame}
\frametitle{Fonctionnement des règles de matching}
\begin{block}{}
\textcolor{red}{alert} http any any $\rightarrow$ any any (msg:""; content:"inria.fr";)
\end{block}
Actions :
\begin{enumerate}
\item pass
\item drop
\item reject
\item alert
\end{enumerate}
\end{frame}

%7 eme page : Exemple de regles (http)
\begin{frame}
\frametitle{Fonctionnement des règles de matching}
\begin{block}{}
alert \textcolor{green}{http} any any $\rightarrow$ any any (msg:""; content:"inria.fr";)
\end{block}
Protocole :
\begin{itemize}
\setbeamertemplate{itemize item}[triangle]
\item tcp / udp
\item ip
\item icmp
\end{itemize}
\end{frame}

%8 eme page : Exemple de regles (Source / Destination)
\begin{frame}
\frametitle{Fonctionnement des règles de matching}
\begin{block}{}
alert http \textcolor{blue}{any any}  $\rightarrow$ \textcolor{blue}{any any} (msg:""; content:"inria.fr";)
\end{block}
Source/Destination Port :
\begin{itemize}
\setbeamertemplate{itemize item}[triangle]
\item 128.93.162.84 80 $\rightarrow$ 192.168.17.218 any
\item $\$EXTERNAL\_NET$ any <> $\$HOME\_NET$ any
\end{itemize}
\end{frame}

%9 eme page : Exemple de regles (Motif)
\begin{frame}
\frametitle{Fonctionnement des règles de matching}
\begin{block}{}
alert http any any $\rightarrow$ any any (msg:""; \textcolor{cyan}{content:"inria.fr";})
\end{block}
\begin{center}
Motif
\end{center}
\end{frame}

%10 eme page : Exemple de regles (autre parametres)
\begin{frame}
\frametitle{Fonctionnement des règles de matching}
\begin{block}{}
alert http any any $\rightarrow$ any any (\textcolor{cyan}{msg:"";} content:"inria.fr";)
\end{block}
Autres paramètres :
\begin{itemize}
\setbeamertemplate{itemize item}[triangle]
\item msg:"Connexion établie depuis le site www.inria.fr"
\item http\_uri, http\_method, http\_header, http\_cookie $\ldots$
\item flow:established,to\_server; to\_client; nocase; $\ldots$
\end{itemize}
\end{frame}

% Partie III
%11 eme page : Flowint
\section{Fonctionnalité avancée}
\subsection{Flowint}
\begin{frame}
\frametitle{Flowint}
\begin{block}{Initialisation d'une variable}
alert tcp any any $ \Rightarrow $ any any (msg: "Start a login count"; content: "login failed"; flowint: loginfailed, notset; flowint: loginfail, =, 1; sid:999997; rev:5;)
\end{block}
\pause
\begin{block}{Incrémentation d'une variable}
alert tcp any any $ \Rightarrow $ any any (msg: "Counting Logins"; content: "login failed"; flowint: loginfailed, isset; flowint: loginfail, +, 1;)
\end{block}
\end{frame}


%12 eme page : Fonctionnalitée avancée. LibHtp
\subsection{libhtp}
\begin{frame}
\frametitle{Suricata et LibHtp}
Capable de décoder des flux compressés par Gzip
\begin{block}{Page non compressée}
alert http 128.93.162.84 any -> any any (msg:"LOCAL Flux depuis inria.fr mot clé (http)"; flow:to\_client; content:"Inria recrute"; nocase; sid:999992; rev:5;)
\end{block}
\begin{block}{Page compressée}
alert http any any -> any any (msg:"LOCAL Flux depuis UGA mot clé (http)"; flow:to\_client; content:"est astrophysicien"; http\_server\_body; nocase; sid:999990; rev:5;)
\end{block}
\end{frame}

%13 eme page : Décompression de fichiers
\subsection{Fichiers}
\begin{frame}
\frametitle{Décompression de fichiers}
Extraction et inspection des fichiers compressé.
\begin{block}{Règle de base}
alert http any any -> any any (msg:"FILE store all"; filestore; sid:1; rev:1;)
\end{block}
\end{frame}



\end{document}