\documentclass{beamer}

\usepackage[frenchb]{babel}
\usepackage[T1]{fontenc}
\usepackage[utf8]{inputenc}

\usetheme{Warsaw}

\title{Untraceable Electronic Mail, Return Addresses, and Digital Pseudonyms - By David Chaum}
\author{Présenté par Aurélien Monnet-Paquet}
\institute{www.inria.fr}
\date{19 Mai 2016}

\begin{document}

%Premiere page : page de garde.
\begin{frame}
\titlepage
\end{frame}

%Partie I : Introduction
%Deuxieme page : Introduction
\section{Introduction}
\subsection{L'analyse du trafic réseaux}
\begin{frame}
\frametitle{Problème ?}
\begin{itemize}
\setbeamertemplate{itemize item}[triangle]
\item Comment garder confidentiel qui parle avec qui et quand ?
\pause
\item Cryptographie à clé publique
\end{itemize}
\end{frame}

%Notations
\subsection{Notation utilisée}
\begin{frame}
\frametitle{Notation}
\begin{itemize}
\setbeamertemplate{itemize item}[triangle]
\item K $~~~~~~~$: Clé publique
\item Inv(K) $~$: Clé privée
\item X $~~~~~~~$: Message en clair
\pause
\item $Inv(K)( K( X ) ) = K( Inv(K)( X ) ) = X$
\pause
\item R $~~~~~~~$: Une chaîne de bits aléatoire
\pause
\item Le chiffrement du message par la clé publique :
\begin{center}
\item $K( R, X )$
\end{center}
\end{itemize}
\end{frame}

%Troisieme page : Hypothèses
\subsection{Hypothèses}
\begin{frame}
\frametitle{Hypothèses}
\begin{block}{Hypothèse 1}
Personne ne peut déterminer quoi que ce soit sur les correspondances entre un ensemble d'éléments chiffrés et l'ensemble des éléments non chiffrés, ou créer des contrefaçons sans la chaîne aléatoire appropriée ou la clé privée.
\end{block}
\pause
\begin{block}{Hypothèse 2}
Chaque serveurs connaît le serveur d'origine du message qu'il reçoit, ainsi que le/les destinataires. Chaque serveurs peuvent injecter, modifier, supprimer des messages.
\end{block}

\end{frame}

%----------------------------------------------------------------------------------------
%Partie II : Mail system
% Mail system
\section{Mail system}
%6 eme page : 
\subsection{Fonctionnement}
\begin{frame}
\frametitle{Mail system}
Alice souhaite envoyer un message à Bob :
\begin{itemize}
\setbeamertemplate{itemize item}[triangle]
\item Alice chiffre le message M avec la clé publique (Ka) de Bob
\item Alice ajoute l'adresse (A) de Bob au résultat
\item Alice chiffre le tout avec la clé publique (K1) du premier serveur de mix
\end{itemize}
\end{frame}

\begin{frame}
%\frametitle{Alice et Bob}
Alice souhaite envoyer un message à Bob :
\begin{itemize}
\setbeamertemplate{itemize item}[triangle]
\item Alice \textcolor{red}{chiffre} le message \textcolor{red}{M} avec la clé publique \textcolor{red}{(Ka)} de Bob
\item Alice ajoute l'adresse (A) de Bob au résultat
\item Alice chiffre le tout avec la clé publique (K1) du premier serveur de mix
\end{itemize}
\begin{center}
$K1( R1, $\textcolor{red}{$Ka( R0, M )$}$, A ) \rightarrow Ka( R0, M ), A.$
\end{center}
\end{frame}

\begin{frame}
%\frametitle{Alice et Bob}
Alice souhaite envoyer un message à Bob :
\begin{itemize}
\setbeamertemplate{itemize item}[triangle]
\item Alice chiffre le message M avec la clé publique (Ka) de Bob
\item Alice \textcolor{blue}{ajoute} l'adresse \textcolor{blue}{(A)} de Bob au résultat
\item Alice chiffre le tout avec la clé publique (K1) du premier serveur de mix
\end{itemize}
\begin{center}
$K1( $\textcolor{blue}{$R1, $}$Ka( R0, M )$\textcolor{blue}{$, A$}$ ) \rightarrow Ka( R0, M ), A.$
\end{center}
\end{frame}

\begin{frame}
%\frametitle{Alice et Bob}
\begin{center}
$K1( R1, Ka( R0, M ), A ) \rightarrow \textcolor{blue}{Ka( R0, M ), A}.$
\end{center}

\begin{itemize}
\setbeamertemplate{itemize item}[triangle]
\item Le serveur déchiffre ce qu'il reçoit avec sa clé privée
\item Puis enlève la chaîne R1 et délivre le \textcolor{blue}{reste}
\end{itemize}
\end{frame}

\subsection{Serveur de mail anonyme}
\begin{frame}
%\frametitle{Alice et Bob}
But d'un tel serveur :
\begin{itemize}
\setbeamertemplate{itemize item}[triangle]
\item Cacher les correspondances entre les éléments en entrée/sortie
\item Cacher l'ordre d'arriver en fournissant des éléments de taille uniforme en sortie
\item Traiter une et une seule fois un lot
\end{itemize}
Si R1 contient un horodatage, le serveur ne conserve pas de copie des lots
\end{frame}

\begin{frame}
%\frametitle{Alice et Bob}
Signature
\end{frame}

\begin{frame}
%\frametitle{Alice et Bob}
Cascade
\end{frame}


%--------------------------------------------------------------------------------------
% Partie III : Return addresses
% Return addresses
\section{Return addresses}
\subsection{Comment répondre ?}
\begin{frame}
\frametitle{Comment répondre en gardant l'adresse d'Alice anonyme ?}
Former une adresse anonyme de retour :
\begin{center}
K1(R1, \textcolor{blue}{Ax}), Kx
\end{center}
\begin{itemize}
\setbeamertemplate{itemize item}[triangle]
\item \textcolor{blue}{Ax} est l'adresse réelle d'Alice
\end{itemize}
\end{frame}

\begin{frame}
\frametitle{Comment répondre en gardant l'adresse d'Alice anonyme ?}
Former une adresse anonyme de retour :
\begin{center}
K1(R1, Ax), \textcolor{red}{Kx}
\end{center}
\begin{itemize}
\setbeamertemplate{itemize item}[triangle]
\item \textcolor{blue}{Ax} est l'adresse réelle d'Alice
\item \textcolor{red}{Kx} est une clé publique choisie pour l'occasion
\end{itemize}
\end{frame}

\begin{frame}
\frametitle{Comment répondre en gardant l'adresse d'Alice anonyme ?}
Former une adresse anonyme de retour :
\begin{center}
K1(\textcolor{green}{R1}, Ax), Kx
\end{center}
\begin{itemize}
\setbeamertemplate{itemize item}[triangle]
\item \textcolor{blue}{Ax} est l'adresse réelle d'Alice
\item \textcolor{red}{Kx} est une clé publique choisie pour l'occasion
\item \textcolor{green}{R1} est une clé / chaîne aléatoire à des fins d'étanchéité
\end{itemize}
\end{frame}

\subsection{Nouveau type de messages}
\begin{frame}
%\frametitle{}
\begin{center}
$K1( R1, Ax ), \textcolor{blue}{Kx( R0, M )} \rightarrow Ax, R1( Kx( R0, M ))$
\end{center}
\begin{itemize}
\setbeamertemplate{itemize item}[triangle]
\item \textcolor{blue}{$Kx( R0, M )$} est 
\end{itemize}
\end{frame}

\begin{frame}
%\frametitle{}
\begin{center}
$\textcolor{red}{K1( R1, Ax )}, Kx( R0, M ) \rightarrow Ax, R1( Kx( R0, M ))$
\end{center}
\begin{itemize}
\setbeamertemplate{itemize item}[triangle]
\item \textcolor{blue}{$Kx( R0, M )$} est 
\item \textcolor{red}{$K1( R1, Ax )$} est le chiffrement de l'adresse d'Alice + la clé R1 avec l'adresse publique du premier serveur
\end{itemize}
\end{frame}

\begin{frame}
%\frametitle{}
\begin{center}
$K1( \textcolor{green}{R1}, Ax ), Kx( R0, M ) \rightarrow Ax, \textcolor{green}{R1}( Kx( R0, M ))$
\end{center}
\begin{itemize}
\setbeamertemplate{itemize item}[triangle]
\item \textcolor{blue}{$Kx( R0, M )$} est 
\item \textcolor{red}{$K1( R1, Ax )$} est le chiffrement de l'adresse d'Alice + la clé R1
 avec l'adresse publique du premier serveur
\item \textcolor{green}{$R1$} est utilisé pour re-chiffré $Kx( R0, M )$
\end{itemize}
\end{frame}

\subsection{En cascade}
\begin{frame}
\frametitle{En utilisant une cascade de serveurs}
\begin{center}
$K1( R1, K2( R2, ..., \textcolor{blue}{A} ...)), Kx( R0, M )$
\end{center}
\begin{itemize}
\setbeamertemplate{itemize item}[triangle]
\item Avec A = \textcolor{blue}{$K<n-1>( R<n-1>, Kn( Rn, Ax ))$}
\end{itemize}
\end{frame}

\subsection{Accusé de réception}
\begin{frame}
%\frametitle{}
Alice peut recevoir un accusé de réception sur le message qu'elle envoie à Bob :
\begin{itemize}
\setbeamertemplate{itemize item}[triangle]
\item L'adresse réelle d'Alice est étendu avec une autre adresse anonyme (pour Alice et Bob)
\pause
\item Envoi de l'accusé une fois que le dernier serveur a envoyé le message
\pause
	\begin{itemize}
	\item Adresse de livraison du message
	\item Le message
	\item Peut être signé par tous les serveurs intermédiaire
	\end{itemize}
\end{itemize}
\end{frame}

%-------------------------------------------------------------------------------------
% Partie IV : Digital Pseudonyms
% Digital Pseudonyms
\section{Digital Pseudonyms}
\subsection{Définition}
\begin{frame}
\frametitle{Digital Pseudonyms}
\begin{block}{Définition}
Un "pseudonyme" numérique est une clé publique utilisée pour vérifier les signatures effectuées par le titulaire anonyme de la clé privée correspondante.
\end{block}
\pause
\begin{block}{Comment ?}
Une autorité est chargé de tenir à jour une liste de pseudonymes.
\end{block}
\pause
\begin{block}{Pourquoi ?}
Dialoguer avec une entité de manière anonyme tout en garantissant qu'on utilise pas plusieurs identités.
\end{block}
\end{frame}

\subsection{Faire une demande}
\begin{frame}
%\frametitle{1}
Chaque demande d'acceptation :
\begin{itemize}
\setbeamertemplate{itemize item}[triangle]
\item Peut être soumis par le system de mail anonyme
\item Doit contenir la clé publique et le pseudonyme souhaité
\item Peut être pour une liste particulière soumise à un vote
\item Peut être acceptée ou refusée par une autorité
\end{itemize}
Une liste contenant les couples (pseudonyme, clé publique) est publié par cette autorité  
\end{frame}

\subsection{Élection}
\begin{frame}
%\frametitle{1}
L'acceptation dans une liste particulière peut être soumise a un vote :
\begin{itemize}
\setbeamertemplate{itemize item}[triangle]
\item Les autres membres de cette liste votent pour/contre
\item Un vote est de la forme : $K1( R1, K, Inv(K)( C, V ))$
\pause
	\begin{itemize}
		\item K = pseudonyme du votant
		\item V son vote
		\item Comptabilisé après avoir vérifié le pseudonyme du votant et la signature du vote
	\end{itemize}
\end{itemize}
Une liste contenant les couples (pseudonyme, clé publique) est publié par cette autorité  
\end{frame}


%-------------------------------------------------------------------------------------
% Partie V : General Purpose Mail Systems
% General Purpose Mail Systems
\section{General Purpose Mail Systems}
\subsection{Généralités}
\begin{frame}
\frametitle{General Purpose Mail Systems}
\begin{itemize}
\setbeamertemplate{itemize item}[triangle]
\item Chaque message devrait passer par une "cascade" de serveurs
\item Les serveurs peuvent fonctionner de manière continue ou périodique
\item Les messages (trop) longs sont divisés en plusieurs parties :
	\begin{itemize}
		\item $Ka( R0, M ) = M1, M2, ..., M<l-n>$
	\end{itemize}
\end{itemize}
\end{frame}

\subsection{Alléger le coût}
\begin{frame}
%\frametitle{1}
Un message qui passe par tous les serveurs intermédiaire possible a un coût non négligeable
\begin{itemize}
\setbeamertemplate{itemize item}[triangle]
\item Manière 1 : sous-ensemble
\item Manière 2 : valeur aléatoire
\item Une séquence de serveur peut être définie pour un message
\end{itemize}
\end{frame}

\begin{frame}
%\frametitle{1}
Détail des formules pour trouver une séquence ?
\end{frame}

\subsection{Conclusion}
\begin{frame}
\frametitle{Conclusion}
Solution efficace pour sécurisé les messages contre l'analyse du trafic réseau.\\
Permet également d'envoyer / recevoir des messages anonyme avec/sans pseudonyme.
\end{frame}

\end{document}