\documentclass{article}           %% ceci est un commentaire (apres le caractere %)
\usepackage[latin1]{inputenc}     %% adapte le style article aux conventions francophones
\usepackage[T1]{fontenc}          %% permet d'utiliser les caractères accentués
\usepackage[dvips]{graphicx}      %% permet d'importer des graphiques au format .EPS (postscript)
\usepackage{fancybox}		   %% package utiliser pour avoir un encadré 3D des images
\usepackage{makeidx}              %% permet de générer un index automatiquement
\usepackage{multirow}
\title{}     %% \title est une macro, entre { } figure son premier argument
\author{}        %% idem

\makeindex		    %% macro qui permet de générer l'index
\bibliographystyle{prsty}	  %% le style utilisé pour créer la bibliographie
\begin{document}                  %% signale le début du document

Tableau :
\begin{table}[ht!]
	\begin{center}
		\begin{normalsize}
			\begin{tabular}{|l|l|}
    				\hline
     			\textbf{Name} & \textbf{Type}\\
                \hline
                payload & malware \\
    				\hline
    				harmless file & legitimate file\\
    				\hline
    				250Mo.zip & \multirow{ 4}{*}{little boy (size)} \\
    				500Mo.zip &  \\
    				1Go.zip &  \\
    				2Go.zip &  \\
    				\hline
    				r.zip &  quine\\
    				\hline
    				snake.zip &  snake\\
    				\hline
    				42.tar.bz2 & fragmentation \\
    				\hline
 			\end{tabular}
 		\end{normalsize}
	\end{center}
\end{table}

\end{document}