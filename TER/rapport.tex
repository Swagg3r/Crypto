%				PLAN
% 0. Abstract
% 1. Intro
% 2. IDS / Suricata
% 3. Agregateur d'AV
% 4. Framework : mastiff, viper, irma
% 5. Antivirus Linux
% 6. Conclu
%
%
\RequirePackage{fix-cm}
\documentclass[smallextended]{svjour3}       % onecolumn (second format)
\smartqed  
\usepackage{graphicx}
\usepackage[utf8]{inputenc}
\usepackage[T1]{fontenc}

\begin{document}

\title{Étude des fichiers compressé sur les moyens de protection défensifs}
%L'effet des bombes de compression sur les IDS / Antivirus
%\subtitle{Do you have a subtitle?\\ If so, write it here}

%\titlerunning{Short form of title}        % if too long for running head

\author{Monnet-Paquet Aurélien  \\ \and \\
        Supervised by : Lauradoux Cédric
}

%\authorrunning{Short form of author list} % if too long for running head

\institute{Monnet-Paquet Aurelien \at
              Université Grenoble Alpes, 38400 Saint Martin d'Hères \\
              \email{monnetpa@e.ujf-grenoble.fr}
           \and
           Lauradoux Cedric \at
              Inria, 38334 Montbonnot\\
              \email{cedric.lauradoux@inria.fr}
}

\date{Juin 2016}


\maketitle

\begin{abstract}
Insert your abstract here. Include keywords, PACS and mathematical
subject classification numbers as needed.
\keywords{Bombe de compression \and IDS \and Antivirus \and Framework}
% \PACS{PACS code1 \and PACS code2 \and more}
% \subclass{MSC code1 \and MSC code2 \and more}
\end{abstract}

\section{Introduction}
\label{intro}
Your text comes here. Separate text sections with

%-------------------------------------------------------------
%	Partie IDS / Suricata
%-------------------------------------------------------------
\newpage
\section{Les IDS : Sonde de détection d'intrusion}
\label{sec1:ids}
Text with citations \cite{Suricata} and \cite{virustotal}.

\subsection{Présentation}
\label{ids:présentation}
But : faire peter Suricata avec les bombes.

\subsection{Fonctionnement}
\label{ids:fonctionnement}

\subsection{Analyse}
\label{ids:analyse}

\subsection{Conclusion sur les IDS}
\label{ids:conclusion}

%-------------------------------------------------------------
%	Partie Antivirus / agregateurs
%-------------------------------------------------------------
\newpage
\section{Les sites web agrégateurs d'Antivirus}
\label{sec2:agrégateurs}
%1 ere vague d'analyse avec le payload normal (base)
%2 eme vague avec le payload en mode compressé pour analyser le taux de detection
%3 eme vague les bombes de compressions.

%Analyses des différences entre les même AV des différentes plates-formes.

\subsection{Présentation}
\label{agrégateurs:présentation}
Dans cette section, nous allons analyser le comportement des agrégateurs d'antivirus que l'on peut trouver sur internet.\\
Cette expérience portera sur 3 sites : 
\begin{itemize}
\item VirusTotal \cite{virustotal}
\item Jotti \cite{Jotti}
\item Virscan \cite{Virscan}
\end{itemize}
Nous allons tester chacun de ses sites en 3 étapes :
\begin{itemize}
\item 1. Analyse témoin : un fichier connu pour être malveillant
\item 2. Analyse du même fichier dans différents formats de compression
\item 3. Analyse des différentes bombes de compression
\end{itemize}
Le but de cette expérience est d'analyser le comportement de ses éléments sur les fichiers compressés.\\
Est-ce que tous les format de compression sont supporté ?\\
Le comportement des antivirus commun entre les sites sont ils similaire ?\\
Est-ce que les bombes sont détectées ?\\

\subsection{Fonctionnement}
\label{agrégateurs:fonctionnement}
VirusTotal est actuellement le site le plus populaire dans ce domaine. Il permet, à ce jour, d'analyser un fichier avec 57 antivirus différents. Il existe 3 manières pour soumettre un fichier :
\begin{itemize}
\item L'interface web
\item L'API
\item Recherche par hash (md5, sha-1, sha-256) 
\end{itemize}
Remarque : pour pouvoir utilisé l'API, il faut avoir un compte actif.
\subsection{Analyse}
\label{agrégateurs:analyse}

\subsection{Conclusion sur les agrégateurs}
\label{agrégateurs:conclusion}

%-------------------------------------------------------------
%	Partie Framework : mastiff / viper / irma ?
%-------------------------------------------------------------
\newpage
\section{Les Frameworks d'analyse de malwares}
\label{sec3:frameworks}

\subsection{Présentation}
\label{frameworks:présentation}

\subsection{Fonctionnement}
\label{frameworks:fonctionnement}

\subsection{Analyse}
\label{frameworks:analyse}

\subsection{Conclusion sur les frameworks}
\label{frameworks:conclusion}

%-------------------------------------------------------------
%	Partie AV linux --
%-------------------------------------------------------------
\newpage
\section{Les Antivirus Linux}
\label{sec4:av}

\subsection{Présentation}
\label{av:présentation}

\subsection{Fonctionnement}
\label{av:fonctionnement}

\subsection{Analyse}
\label{av:analyse}

\subsection{Conclusion sur les frameworks}
\label{av:conclusion}



%-------------------------------------------------------------
%	Partie Conclusion
%-------------------------------------------------------------
\newpage
\section{Conclusion}
\label{sec5:conclusion}

%\begin{acknowledgements}
%If you'd like to thank anyone, place your comments here
%and remove the percent signs.
%\end{acknowledgements}

% BibTeX users please use one of
%\bibliographystyle{spbasic}      % basic style, author-year citations
%\bibliographystyle{spmpsci}      % mathematics and physical sciences
%\bibliographystyle{spphys}       % APS-like style for physics
%\bibliography{}   % name your BibTeX data base

% Non-BibTeX users please use
\newpage
\begin{thebibliography}{}
%
% and use \bibitem to create references. Consult the Instructions
% for authors for reference list style.

\bibitem{PerformanceAnalysis}
David J. Day et Benjamin M. Burns, "A Performance Analysis of Snort and Suricata Network Intrusion Detection and Prevention Engines"

\bibitem{Suricata}
oisf.net, The Open Information Security Foundation, organisation à but non lucrative qui développe et met à jour Suricata.

\bibitem{SuricataDoc}
redmine.openinfosecfoundation.org/projects/suricata/wiki/, la documentation pour utilisateurs et développeurs de Suricata.

\bibitem{virustotal}
virustotal.com, site agrégateurs d'antivirus, filière de Google.

\bibitem{Jotti}
virusscan.jotti.org, site agrégateurs d'antivirus, basé aux Pays-Bas.

\bibitem{Virscan}
www.virscan.org, site agrégateurs d'antivirus, basé en chine.

\bibitem{Mastiff}
www.korelogic.com

\bibitem{Viper}
www.viper.li

\bibitem{Irma}
irma.quarkslab.com

\bibitem{Comodo}
www.comodo.com

\bibitem{Clamav}
www.clamav.net

%
%http://www.av-comparatives.org/wp-content/uploads/2015/05/avc_linux_2015_en.pdf

\end{thebibliography}

\end{document}
% end of file template.tex

