%				PLAN
% 0. Abstract
% 1. Intro
% 2. Outils de sécurité
% 	2.1 AV
%	2.2 Aggregateurs
%		2.2.1 Site web
%		2.2.2 Framework
%	2.3 IDS
% 3. Test
%	3.1 Protocole de test
%	3.2 Resultats et observations
% 4. Conclu

\RequirePackage{fix-cm}
\documentclass[smallextended]{svjour3}       % onecolumn (second format)
\smartqed  
\usepackage{graphicx}
\usepackage{hyperref}
\usepackage[utf8]{inputenc}
\usepackage[T1]{fontenc}

\begin{document}

\title{La sécurité des outils de sécurité}
%L'effet des bombes de compression sur les IDS / Antivirus
%\subtitle{Do you have a subtitle?\\ If so, write it here}

%\titlerunning{Short form of title}        % if too long for running head

\author{Monnet-Paquet Aurélien  \\ \and \\
        Supervised by : Lauradoux Cédric
}

%\authorrunning{Short form of author list} % if too long for running head

\institute{Monnet-Paquet Aurelien \at
              Université Grenoble Alpes, 38400 Saint Martin d'Hères \\
              \email{monnetpa@e.ujf-grenoble.fr}
}

\date{Juin 2016}


\maketitle

\begin{abstract}
$ $\\
%Présentation de l'étude du comportement des outils défensif appliqué à un environnement de fichiers compressé par différents types d'algorithmes.\\
%Cette étude permettra d'analyser le comportement des outils lorsqu'ils sont soumis à des fichiers compressés.

\keywords{Bombe de compression \and IDS \and Antivirus \and Framework}
\end{abstract}

%-------------------------------------------------------------
%	INTRODUCTION
%-------------------------------------------------------------
\section{Introduction}
\label{intro}
%Selon l'étude de McAfee/CSIS de 2014, les pertes économique dues à la cybercriminalité représente 445 milliards de dollars par an, avec des attaques en forte hausse.\\
%Le but de ce rapport est d'analyser le comportement des outils défensifs sur les fichiers compressés.\\
%Est-ce que tous les formats de compression sont supportés ?\\
%Le comportement des antivirus commun entre les sites sont-ils similaire ?\\
%Est-ce que les bombes sont détectées ?\\

%-------------------------------------------------------------
%	OUTILS DE SECURITE
%-------------------------------------------------------------
\section{Outils de sécurité}
\label{2.Outils}

\subsection{Antivirus}
\label{2.1antivirus}
Les antivirus (AV) sont des programmes très importants en sécurité. Ce sont des programmes installés sur les machines des utilisateurs et ils permettent de détecter les malwares et les empêchent de s’exécuter sur le système. Il existe différentes manières de détecter un malware : 
\begin{itemize}
\item Scan par signature : On calcule la signature d'un fichier ou d'un morceau de code et on le compare à une base de données. Cette méthode est inefficace contre les malwares polymorphes ou capable de changer leur signature. C'est cependant la méthode la plus populaire chez les concepteurs d'antivirus.
\item Analyse heuristique : La plus puissante des méthodes car elle permet de simuler l’exécution du code d'un programme dans une zone contrôlée. Ainsi, l'AV peut observer le comportement du code qui s’exécute et définir si il s'agit d'un malware ou non. Cette méthode peut provoquer des fausses alertes et coûte du temps CPU.
\item Contrôle d'intégrité : Permet de vérifier qu'un fichier n'a pas été modifié au cours du temps. Les informations comme la taille, la date et l'heure de dernière modification, la somme de contrôle éventuelle du fichier sont analysées lors de la demande d'ouverture du fichier par l'utilisateur (si analyse en temps réel) ou lors d'un scan de l'AV.\\
Le but de notre expérience est de tester s'il est possible de faire un déni de service lorsque l'AV analyse une bombe de compression.\\
Nous avons testé deux AV pour Linux : ClamAV (\url{www.clamav.net}) et Comodo (\url{www.comodo.com})\\
\end{itemize}

\subsection{Agrégateurs d'antivirus}
\label{2.2agrégateurs}
Les agrégateurs d'AV sont des outils qui permettent d'analyser des fichiers avec plusieurs AV en même temps. Le résultat obtenu en analysant un fichier avec ce type d'outils est plus fiable qu'avec un seul AV. Cependant, cet outil ne doit pas remplacer un AV installé sur une machine. Il existe deux types d’agrégateurs :  

\subsubsection{Sites web}
\label{2.2.1sites}
Ces sites web spécialisés permettent d'analyser les fichiers grâce à des dizaines d'antivirus différents. Un partenariat relie les éditeurs d'AV et le site. Les éditeurs d'AV mettent à disposition des sites leur programme. Cela signifie que les éditeurs sont responsables de leur configuration. Il peut arriver qu'un AV soit configuré sur un site différemment que la version commerciale. En effet, il s'agit d'un bon moyen pour les éditeurs de tester de nouvelles fonctionnalités, configurations. Lorsqu'un fichier est soumis par un utilisateur sur cette plateforme, il est analysé par les moteurs des différents AV. Dans le cas où au moins un AV détecte un fichier comme malveillant, le fichier est envoyé aux éditeurs des AV qui ne l'on pas détecter pour l'améliorer. Nous avons choisis 3 sites agrégateurs d'AV pour réaliser nos tests :\\
\begin{tabular}{|l|p{3cm}|p{3cm}|p{3cm}|}
  \hline
   \textbf{Site web} & \textbf{VirusTotal} & \textbf{Jotti} & \textbf{Virscan} \\
   \hline
  Nombre d'antivirus & 57 & 19 & 39\\
  \hline
  Situation géographique & Filiale de Google, basée en espagne & organisation basée aux pays-bas & organisation basée en chine\\
  \hline
  Type d'hebergement & Cloud & Cloud & Cloud\\
  \hline
\end{tabular}
$ $\\
\textit{Remarque} : VirusTotal est le plus populaire avec en moyenne 1.5M de soumissions par jours (dont 16.6\% provenant de France).\\
$ $\\
Pour utiliser ce service, un utilisateur doit envoyer le fichier suspect vers le site pour que les moteurs des différents AV puissent l'analyser. Après analyse, le fichier n'est pas détruit s'il est considéré comme dangereux, car il est envoyé aux éditeurs d'AV pour améliorer leurs programmes. Il n'est donc pas concevable pour une personne (physique ou morale) d'envoyer des fichiers avec des données sensibles vers des serveurs dont cette personne n'a pas de contrôle. C'est pourquoi il existe aussi des « Framework de détection de malwares ».

\subsubsection{Frameworks}
Ce type d'outils fonctionne en local, idéalement sur une machine virtuelle pour éviter de contaminer la machine hôte. Ce sont des programmes qui permettent à un utilisateur d'analyser un fichier suspect de différentes façons via des plugins. Pour notre étude, nous avons utilisé deux \textit{Frameworks} : 
\label{2.2.2frameworks}
\begin{itemize}
\item Mastiff, \url{www.korelogic.com}, société basée au USA, composée de professionnels en sécurité informatique.
\item Viper, \url{www.viper.li}, créer par Claudio Guarnieri, chercheur en sécurité informatique.
\end{itemize}
Ses deux outils sont préinstaller dans la distribution REMnux (\url{remnux.org}) qu'il est facile d'importer dans Virtualbox (\url{www.virtualbox.org}).\\
Mastiff et Viper permettent entre autre:
\begin{itemize}
\item D’identifier et classer le type des fichiers analysé grâce aux règles de YARA, (\url{plusvic.github.io/yara}).
\item Reconnaître deux fichiers dont le contenu est très similaire grâce au fuzzy hashing (ssdeep) (\url{ssdeep.sourceforge.net}). Prenons par exemple les commandes suivantes : "echo message1 > fichier1.txt" et "echo message2 > fichier2.txt". Les deux fichiers générer auront des digests très différents avec aucune ressemblance possible. Cependant, grâce à ssdeep il est possible de faire ressortir le fait que ses deux fichiers sont presque identiques. Technique pratique pour repérer les malwares polymorphes.
\item Extraire et analyser récursivement des fichiers zip. C'est notamment cette dernière qui nous intéresse et que nous allons tester.
\end{itemize}

\subsection{IDS}
\label{2.3ids}
Les systèmes de détection d'intrusion (IDS) sont des programmes permettant d'analyser le trafic réseau de manière transparente. Idéalement l'IDS est situé au point d'entrer du réseau d'une entreprise pour pouvoir analyser tout le trafic réseau. L'IDS fonctionne sur un système de règles et de pattern matching. Un administrateur définit un certain nombre de règles que l'IDS doit vérifier lors de l'analyse du trafic réseau. Suivant les règles que l’administrateur a définies, l'IDS permet :
\begin{itemize}
\item De laisser passer un paquet sans finir l'analyse.
\item Rejeter le paquet, en générant des paquets ICMP erreur ainsi qu'une alerte
\item Générer une alerte dans le fichier de logs.
\end{itemize} 
Par exemple, l'administrateur peut décider de créer une règle  qui générera une alerte lors qu'une connexion sera établie sur les serveurs de Facebook. Cette alerte permet d'identifier l'adresse IP source de la requête HTTP.\\
Le pattern matching fonctionne aussi avec le contenu des paquets qui circulent sur le réseau. De cette manière une règle peut être créée pour générer une alerte lorsqu'un paquet contient les caractères "Euros 2016" par exemple.\\
L'IDS que nous avons choisis de tester est Suricata. Suricata se démarque de la concurrence par une analyse multithread. Avec toujours de plus en plus de données, l'analyse rapide et efficace devient de plus en plus important. L'article de David J. Day et Benjamin M. Burns \cite{PerformanceAnalysis}  analyse les performances de Suricata. Suricata supporte nativement l'IPv6 et est open source. Suricata extrait les fichiers compressés pour analyser leur contenu \cite{SuricataExtraction}. Le but de l’étude est de vérifier qu'il n'y a pas de déni de service possible lorsque Suricata analyse une bombe de compression.


%-------------------------------------------------------------
%	TESTS
%-------------------------------------------------------------
\section{Tests}
\label{3.Tests}

\subsection{Protocole de test}
\label{3.1Protocole}
Dans un premier temps, nous avons généré un exécutable Windows (payload via metasploit) connu pour être analysé comme un malware par la plupart des outils disponibles sur le marché. Ensuite, pour chacun des outils, nous avons effectué une analyse témoin. Cette analyse se compose du fichier générer précédemment, et d'un fichier contenant uniquement des zéros et qui est donc complètement inoffensif.\\
Dans un second temps, nous avons analysé les résultats des deux premiers fichiers compressés dans différents formats : .gz, .lzma, .tar.bz2, .tar.gz, .tar.xz, .lz, .lzo, .zip pour tenter de leurrer les AV. Une autre méthode de contournement d'AV est décrite dans l'article "Contournement d'antivirus par génération d'attaques caméléon"~\cite{Contournement}.\\
Enfin, nous avons utilisé des fichiers dans le but de provoquer un déni de service, voire de provoquer une erreur lors de l'exécution des AV. Ces fichiers sont des bombes de compression : des fichiers qui à la décompression engendre des fichiers de très grande taille.\\
$ $\\
Nous avons réalisé les tests (et installer les outils nécessaires) sur une machine ayant un OS Linux (Ubuntu 12.04 LTS).\\
Après avoir installé ClamAV (via les dépôts Ubuntu/Debian) nous avons effectué des tests dans sa configuration initiale. Cependant, deux paramètres sont à prendre en compte pour mener à bien cette expérience :
\begin{itemize}
\item "max-recursion= $n$" : la profondeur maximale d'exploration lors d'une décompression, 16 par défaut.
\item "max-filesize= $n$" : Extrait et analyse  n octets de chaque archives. 25 Mo par défaut avec une limite de 4 Go.
\end{itemize}
$ $\\
Nous avons également installé un second antivirus : Comodo (via le package du site officiel). Il dispose d'une interface graphique, mais ne nous permet pas de modifier des paramètres comme la récursion, ou la taille des fichiers analysés. Pour lancer un scan, il faut suivre les indications sur l'interface graphique.\\
$ $\\
VirusTotal met à notre disposition une API permettant d'envoyer des fichiers pour analyse via un script. Nous avons alors repris puis modifier un script existant pour effectuer nos tests. Ce script (écrit en Perl) effectue deux requêtes vers VirusTotal. Une première (HTTP POST) pour envoyer le fichier suspect. Et une seconde, pour récupérer le résultat sous forme d'un objet JSON. Ensuite le script génère un rapport mis en forme. Ce rapport contient le résultat des antivirus partenaires de VirusTotal.\\
$ $\\
En ce qui concerne Jotti et Virscan, nous avons utilisé l'interface web pour soumettre nos fichiers et récupérer les résultats.\\
$ $\\
Mastiff est un programme dont le processus peut être exécuté dans un docker. Dans ce cas, un dossier est partagé entre la VM et le docker. Dans ce dossier, nous avons au préalable placé les fichiers à analyser par Mastiff. Dans le cas présent, nous n'avons pas de fichier confidentiel, c'est pourquoi nous avons ajouté l'option d'envoyer les fichiers à analyser vers VirusTotal automatiquement. Lors de son analyse, Mastiff génère un fichier de résultats par moteur d'analyse.\\
$ $\\
Viper est aussi un programme dont le processus peut s’exécuter dans un docker. A la différence que Viper met à notre disposition une interface web, à partir de laquelle nous pouvons soumettre nos fichiers. Une fois un fichier soumis, une multitude de commandes peuvent être lancées pour tester le fichier suspect. Les résultats de ses commandes se retrouvent sur cette interface web.\\
$ $\\
Après avoir installé Suricata, nous avons rédigé des règles pour que les fichiers soient décompressés puis inspectés. Nous avons pris soin de modifier la configuration initiale de Suricata pour que l'extraction se passe de manière optimale :
\begin{itemize}
\item stream.reassembly.depth = 4gb, après avoir réassemblé le flux TCP, le fichier ne doit pas dépasser une taille de 4Go.
\item request\_body\_limit = 0 (infini), valeur que le corps de la requête HTTP ne peut pas dépasser.
\item response\_body\_limit = 0 (infini), valeur que le corps de la réponse HTTP ne peut pas dépasser.
\end{itemize}
De plus, sur une autre machine nous avons installé un serveur web disposant des fichiers à tester par Suricata. Ainsi, avec une simple requête sur le serveur, Suricata reconstitue le fichier puis l'inspecte.

\subsection{Résultats et observations}
\label{3.2résultats}
\textbf{Résultats} de l'analyse de \textbf{ClamAV} et \textbf{Comodo} :\\
$ $\\
\begin{tabular}{|l|l|l|l|l|}
  \hline
  \textbf{Nom du fichier} & \multicolumn{2}{|c|}{\textbf{ClamAV}} & \multicolumn{2}{|c|}{\textbf{Comodo}}\\
  \hline
   & Résultats & Temps & Résultats & Temps \\
  \hline
  \hline
payload.exe & Positif & 9.2s & Positif & 4s\\
	\hline
fichier\_inoffensif & Négatif & 8.9s & Négatif & 4s\\
	\hline
	\hline
250Mo.gz &  Négatif  & 9.6s & Négatif & 8s\\
	\hline
1Go.gz &  Négatif  & 10s & Négatif & 20s\\
    \hline
    \hline
250Mo.lz &  Négatif  & 8.7s & Négatif & 4s\\
	\hline
1Go.lz &  Négatif & 8.2s & Négatif & 4s\\
    \hline
    \hline
250Mo.lzma &  Négatif  &  8.3s & Négatif & 4s\\
	\hline
1Go.lzma &  Négatif  & 8.7s & Négatif & 4s\\
    \hline
    \hline
250Mo.lzo &   Négatif & 8.3s & Négatif & 4s\\
	\hline
1Go.lzo &  Négatif  & 8.6s & Négatif & 4s\\
    \hline
    \hline
250Mo.tar.bz2 &  Négatif & 11.9s & Négatif & 14s\\
	\hline
1Go.tar.bz2 &  Négatif  &  12.5s & Négatif & 54s\\
    \hline
    \hline
250Mo.tar.gz &   Négatif & 9.7s & Négatif & 11s\\
	\hline
1Go.tar.gz &  Négatif  & 9.8s & Négatif & 52s\\
    \hline
    \hline
250Mo.tar.xz &  Négatif  & 10s & Négatif & 4s\\
	\hline
1Go.tar.xz &  Négatif$^1$  &  10s & Négatif & 4s\\
    \hline
    \hline
250Mo.zip &   Négatif & 12s & Négatif & 8s\\
	\hline
1Go.zip &   Négatif & 25.7s & Négatif & 21s\\
    \hline
    \hline
r.zip & Négatif$^2$ & 8.4s & Négatif & 4s\\
	\hline
42.tar.bz2 & Négatif  & 10.9s & Négatif & 6s$^3$\\
	\hline
\end{tabular}
$ $\\
$^1$ : Warning constaté lors de la décompression pour 1Go.tar.xz, "decompress file size exceeds limits - only scanning 104857600 bytes".\\
$^2$ : En forçant environ 400 récursions (au lieu de 16 par défaut), il y a une erreur de segmentation.\\
$^3$ : Limite de récursion atteinte mais non connue. Utilisation du CPU très importante lors de l'analyse d'une sous partie de 42.tar.bz2.\\
Observations : Grâce au temps des analyses, on remarque que certain formats de compression ne sont pas vraiment analysés.
$ $\\
Résultats pour les sites web agrégateurs d'AV :\\
\begin{tabular}{|*{6}{c|}}
    \hline
\textbf{Nom du fichier} & \textbf{Ratio VirusTotal} & \textbf{Ratio Jotti} & \textbf{Ratio Virscan} \\
	\hline
payload.exe & 42/56 & 17/19 & 20/39\\
	\hline
fichier\_inoffensif & 0/56 & 0/19  & 0/39 \\
	\hline
	\hline
250Mo.gz & 0/56 & 1/19 & 2/39 \\
	\hline
1Go.gz &  1/55  & 3/19 & 3/39\\
    \hline
    \hline
250Mo.lz & 0/57 & 0/19 & 0/39\\
	\hline
1Go.lz & 0/56 & 0/19 & 0/39\\
    \hline
    \hline
250Mo.lzma & 0/57 & 0/19 & 0/39\\
	\hline
1Go.lzma & 0/57 & 0/19 & 0/39\\
    \hline
    \hline
250Mo.lzo & 0/56 & 0/19 & 0/39\\
	\hline
1Go.lzo & 0/55 & 0/19 & 0/39\\
    \hline
    \hline
250Mo.tar.bz2 & 0/55 & 4/18 & 1/39\\
	\hline
1Go.tar.bz2 & 1/57 & 5/19 & 2/39\\
    \hline
    \hline
250Mo.tar.gz & 0/57 & 2/19 & 1/39\\
	\hline
1Go.tar.gz & 1/57 & 3/16 & 2/39\\
    \hline
    \hline
250Mo.tar.xz & 0/57 & 2/20 & 0/39\\
	\hline
1Go.tar.xz & 0/57 & 2/20 & 0/39\\
    \hline
    \hline
250Mo.zip & 1/56 & 2/20 & 0/39\\
	\hline
1Go.zip & 2/55 & 3/20 & 0/39\\
    \hline
    \hline
r.zip & 0/54 & 1/18 & 0/39\\
	\hline
42.tar.bz2 & 1/56 & 4/15 & 0/39\\
	\hline
\end{tabular}
$ $\\
\\
Observations :\\
Pour Virscan si un AV n'a pas trouvé de malware en moins de 60 Secondes alors le résultat est : "Rien n'a été trouvé".\\
Tous les antivirus semble ne pas décompresser les format de type .lz, .lzma et .lzo.
$ $\\
\\
\begin{tabular}{|c|c|c|c|}
    \hline
     \textbf{Nom fichier} & \textbf{VirusTotal} & \textbf{Jotti} & \textbf{Virscan} \\
     \hline
    250Mo.gz &  & avast & Fprot, Panda\\
    \hline
    1Go.gz & VBA32 & Arcabit, avast, VBA32 & Fprot, Panda, VBA32\\
    \hline
    \hline
    250Mo.tar.bz2 & & AVG, Arcabit, avast, Sophos & Fprot\\
    \hline
    1Go.tar.bz2 & VBA32 & AVG, Arcabit, avast, sophos, VBA32 & VBA32, Fprot\\
    \hline
    \hline
    250Mo.tar.gz & & Arcabit, avast & Fprot\\
    \hline
    1Go.tar.gz & VBA32 & Aracabit, avast, VBA32 & Fprot, VBA32\\
    \hline
    \hline
    250Mo.tar.xz & & Arcabit, AVG & \\
    \hline
    1Go.tar.xz & & Arcabit, AVG & \\
    \hline
    \hline
    250Mo.zip & VBA32 & avast, VBA32 & \\
    \hline
    1Go.zip & Baidu, VBA32 & Arcabit, avast, VBA32 & \\
    \hline
    r.zip &  & Sophos  & \\
    \hline
    \hline
    42.tar.bz2 & Zillya & Arcabit, avast, AVG, Sophos  & \\
    \hline
\end{tabular}
$ $\\
Observations :\\
Nous pouvons remarquer que les antivirus ne sont pas configurés de la même manière selon les différents sites hormis VBA32. VBA32 semble être cohérent sur tous les sites sur lequel il est testé. En effet, rien de suspect pour VBA32 sur les fichier de 250Mo mais il y a une détection sur les fichiers de 500Mo. Après analyse approfondie : la limite de détection est de 315Mo.\\
ClamAV et Comodo analysent les fichiers sur Virscan et on remarque qu'il n'y a pas de différence entre la version installée et la version online : rien n'est détecté.\\
$ $\\
Résultats sur Mastiff et Viper :\\
$ $\\
Résultats pour Suricata :\\

%-------------------------------------------------------------
%	CONCLUSION
%-------------------------------------------------------------
\section{Conclusion}
\label{2.3ids}

\begin{thebibliography}{}
%
% and use \bibitem to create references. Consult the Instructions
% for authors for reference list style.

\bibitem{PerformanceAnalysis}
David J. Day et Benjamin M. Burns, "A Performance Analysis of Snort and Suricata Network Intrusion Detection and Prevention Engines"

\bibitem{Suricata}
\url{oisf.net}, The Open Information Security Foundation, organisation à but non lucrative qui développe et met à jour Suricata.

\bibitem{SuricataDoc}
\url{redmine.openinfosecfoundation.org/projects/suricata/wiki}, la documentation pour utilisateurs et développeurs de Suricata.

\bibitem{SuricataExtraction}
\url{blog.inliniac.net/2011/11/29/file-extraction-in-suricata/}

\bibitem{API}
\url{perlgems.blogspot.fr/2012/05/using-virustotal-api-v20.html}

\bibitem{Contournement}
Célia Rouis et Mathieu Roudaut, "Contournement d'antivirus par génération d'attaques caméléon", Grenoble INP - Ensimag


\end{thebibliography}

\end{document}